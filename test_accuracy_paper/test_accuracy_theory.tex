
\section{Theory of Heavy Tailed Self Regularization}
\label{sxn:theory}


Let us write the Energy Landscape (or optimization function) for a typical DNN with $L$ layers, with activation functions $h_{l}(\cdot)$, and with weight matrices and b
iases $\mathbf{W}_{l}$ and $\mathbf{b}_{l}$, as follows:
\begin{equation}
E_{DNN}=h_{L}(\mathbf{W}_{L}\times h_{L-1}(\mathbf{W}_{L-1}\times h_{L-2}(\cdots)+\mathbf{b}_{L-1})+\mathbf{b}_{L})  .
\label{eqn:dnn_energy}
\end{equation}
%WLOG,
We imagine training this model on some labeled data $\{d_{i},y_{i}\}\in\mathcal{D}$, using Backprop, by minimizing the loss $\mathcal{L}$
For simplicity, we do not indicate the structural details of the layers (e.g., Dense or not, Convolutions or not, Residual/Skip Connections, etc.). 
Each layer is defined by one or more weight matrices $\mathbf{W}_{L}$, or tensors.

In this study, we only need to  consider Linear and 2D Convolutional (Conv2D) layers. 
For the Linear layers, each $\mathbf{W}_{L}$ is a single $(N\times M)$ (real valued) 2D matrix, where $N\ge M$.
These include Dense or Fully Connected (FC) layers, as well as 1D Convolutional (Conv1D) layers, Attention matrices, etc.
For the Conv2D layers, with a $c\times d$ kernel, $\mathbf{W}_{L}$ is a 4-index Tensor, of the form $(N\times M \times c\times d)$, consisting
of $c\times d$ 2D feature maps of shape $(N\times M)$.   So each Linear layer $l$ gives $n=1$ 2D matrix $\mathbf{W}_{l}$, and each Conv2D layer $l$ gives $n=c\times d$ 2D matrices $\mathbf{W}_{l,i}$. A typical modern DNN may have anywhere between 5 and 500 2D $\mathbf{W}_{l,i}$ layer matrices.

\paragraph{Heavy Tailed Universality}

For any layer weight matrix $\mathbf{W}$, we construct the associated $M\times M$ (uncentered) correlation matrix
\begin{equation}
\mathbf{X} = \dfrac{1}{N}\mathbf{W}^{T}\mathbf{W}  ,
\label{eqn:unc_corr_mat}
\end{equation}
and form the eigenvalue spectrum of $\mathbf{X}$, 
$$
\mathbf{X}\mathbf{v}_{i}=\lambda_{i}\mathbf{v}_{i}
$$
(where we have dropped the $l,i$ indices).

We call the density of eigenvalues $\rho(\lambda)$ the Empirical Spectral Density (ESD).  

Small Neural Networks \charlesX{complete}

The layer matrices in all large, modern DNNs, do not have a scale cut-off, and, instead display scale-invariance.  
For any modern DNN, the ESD of nearly every \ $\mathbf{W}$ layer matrix can be fit to a power law,

$$\rho(\lambda)\sim\lambda^{-\alpha}$$

The power law exponent $\alpha$ lies, $80-90\%$ of the time, in a Universal range between 2 and 4, $\alpha\in[2,4]$.
We have verified this empirical Universality on empirical result on over 10,000 layer matrices $\mathbf{W}$ spanning over 50 pre-trained DNNs.
Of course, there are exceptions, and in any real DNN,  $\alpha$ may range anywhere from $\sim1.5$ to $10$ or higher.  

The power law exponent $\alpha$ is a complexity metric for a weight matrix; it describes how well that matrix encodes the complex correlations in the training data.
So a natural complexity metric for a DNN is to take a weighted average of the power law exponents $\alpha_{l,i}$ for each layer weight matrix $\mathbf{W}_{l,i}$.

$$\hat{\alpha}:=\dfrac{1}{n}\sum_{l,i}b_{l,i}\alpha_{l,i}$$

The smaller $\hat{\alpha}$, the better we expect the DNN to represent the training data. And, presumably, the better the DNN will generalize.
The only question is, what are good weights $b_{l,i}$ ?

It turns out, we can derive the weighted average $\hat{\alpha}$ directly from the more familiar Product Norm.

\charlesX{THE REST OF THIS SECTION is to justify this complexity metric, using the product norm for DNNs.  I have sketeched out some of the math,.  It just needs to be presented clearly AND TO JUSTIFY computing the average log Norm (which is much faster, much simpler)}

\charles{Maybe should can state a theorem and prove it.}

\paragraph{THEOREM:} \emph{The data dependent VC-like complexity of a Deep Neural Network can be expressed a weighted average the of power law exponents describing the empirical spectral density of the layer weight matrices}

\charles{\paragraph{PROOF:...}}

\paragraph{Product Norm Measures of Complexity}

Recently it has been suggested that the complexity of a DNN, $\mathcal{C}$,  can be defined by something akin to a data dependent VC complexity, the product of norms of the layer weight matrices

$$\mathcal{C}\sim\Vert\mathbf{W}_{1}\Vert\times\Vert\mathbf{W}_{2}\Vert\cdots\Vert\mathbf{W}_{L}\Vert$$

where $\Vert\mathbf{W}\Vert$ may be the Frobenius norm or even the L1-norm.  (Here we can use either  $\Vert\mathbf{W}\Vert$ or $\Vert\mathbf{W}\Vert^{2}$ for our complexity metric, which will make more sense below.


 To that end, we consider a log complexity

$$\log\mathcal{C}\sim\log\bigg[\Vert\mathbf{W}_{1}\Vert\times\Vert\mathbf{W}_{2}\Vert\cdots\Vert\mathbf{W}_{L}\Vert\bigg]$$

$$\sim\bigg[\log\Vert\mathbf{W}_{1}\Vert+\log\Vert\mathbf{W}_{2}\Vert\cdots\log\Vert\mathbf{W}_{L}\Vert\bigg]$$

We define the average log norm of the weight matrices as

$$\langle\log\Vert\mathbf{W}\Vert\rangle=\dfrac{1}{N}\sum_{i=1}^{N}\log_{10}\Vert\mathbf{W}_{i}\Vert$$

which we explicitly define in terms of the base-10 log.

\paragraph{Relation to Heavy Tailed Universality}

The Frobenius norm is related to the integral of the ESD, over the range the power law is a good fit. 

$$\Vert\mathbf{W}_{l,i}\Vert^{2}\sim\int_{x_{min}}^{x_{max}}\lambda\rho_{l,i}(\lambda)d\lambda$$

Technically, the power only describes the tail of the ESD,  for the range $\lambda\in[\lambda_{min},\lambda_{max}]$.
  But for most DNN layer matrices, this range covers most the ESD.  So this should be a pretty good approximation.

If we take the log matrix norm (and dropping the $l,i$ subscripts) we get

$$\log\Vert\mathbf{W}\Vert^{2}\approx\log\int_{\lambda_{min}}^{\lambda_{max}}\lambda\rho(\lambda)d\lambda$$
$$=\log\int_{\lambda_{min}}^{\lambda_{max}}\lambda^{1-\alpha}d\lambda$$
$$=\log\left[\dfrac{\lambda^{\alpha}}{\alpha}\right]^{\lambda_{max}}_{\lambda_{min}}$$

Since $\lambda_{min}\sim 0$, we have \charles{check this}

$$\log\Vert\mathbf{W}\Vert^{2}\approx\alpha\log\lambda_{max}$$

Notice that since the power law exponents mostly display Universality empirically, the range $[\lambda_{min},\lambda_{max}]$ is also roughly the same for all matrices, with the minimum eigenvalue is near zero, $\lambda_{min}\sim 0$, and the maximum eigenvalue is  empirically bounded, roughly $\lambda_{max}\sim\mathcal{O}(10^{1}-10^{2})$.  

This gives 

$$\log_{10}\Vert\mathbf{W}\Vert^{2}\approx(1-2)\times\alpha$$


\charles{FINISH WRITEUP AND and numerical results. In particular, we may be off by some scaling factor in $\rho(\lambda)$}
\charlesX{This could all be presented as a theorem--building on work by Hidary and Poggio}

\charles{We need to clean up the math.  I think we could use $\vert\mathbf{W}_{F}\vert$ or $\Vert\mathbf{W}\Vert$  or $\vert\mathbf{W}_{F}\vert^{2}$ and could form the complexity metrix in terms of the produt norm of W, the product norm squared of W, or even the product norm of X.   }.  

$$\log\Vert\mathbf{W}\Vert\rightarrow \sqrt{b\alpha}$$

where the weight factor is $b$ is 

$$b=\log\lambda_{max}\sim(1-2)$$


$$\text{Linear Layer:}\;\;\log\Vert\mathbf{W}_{l}\Vert\rightarrow\sqrt{b_{l}(\alpha_{l})}$$

For the Conv2D layers, we relate the 'Norm' of the 4-index Tensor $\mathbf{W}_{l}$ to the sum of the integrals of the $n=c\times d$ ESDs for each feature map, giving 

$$\text{Conv2D Layer:}\;\;\log\Vert\mathbf{W}_{l}\Vert\rightarrow-\sum_{i}\sqrt{b_{l,j}(\alpha_{i,l})}$$

So in the expression for the product norm for $\log\mathcal{C}$, let us replace each $\log\Vert\mathbf{W}_{l}\Vert$ for layer $l$ with the sum of the power law exponents $\alpha_{l,i}$ for the $n{_l}$ $\mathbf{W}_{l,i}$ layer matrices, and take the average over all $N_{\alpha}$  matrices.  This gives a new, albeit rather crude complexity metric for the entire DNN

\charles{We don't really use this, but we could:}

$$\hat{\alpha}:=\dfrac{1}{N_{\alpha}}\sum_{i,l}b_{i,j}\alpha_{i,l}$$

\charles{Or we could use}

$$\hat{\alpha}:=\dfrac{1}{N_{\alpha}}\sum_{i,l}\sqrt{b_{i,j}\alpha_{i,l}}$$

\charles{The main complication is if we use log norms or we use a sum of $\alpha$.  Plus we can drop the $-1$ but we have to deal with the sign change, which I have to work out}

We can now use $\hat{\alpha}$ to analyze numerous pre-trained DNNs...and the results are indeed surprising.

\paragraph{Numerical Example of Power Law / Norm Relation}

But first, some numerical examples of the relation between $log\Vert\mathbf{W}\Vert$ and $\alpha$
BLAH

BLAH

BLAH


