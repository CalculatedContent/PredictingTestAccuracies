%\vspace{-4mm}
\section{Discussion}
\label{sxn:discussion}
%\vspace{-3mm}

We have presented an \emph{unsupervised} capacity control metric which predicts trends in test accuracies of a trained DNN---without peeking at the test data. 
This complexity metic, $\hat{\alpha}$ of Eqn.~(\ref{eqn:alpha_hat_specific}), is a weighted average of the PL exponents $\alpha$ for each layer weight matrix, where $\alpha$ is defined in the recent HT-SR Theory~\cite{MM17_TR,MM18_TR}, and where the weights are the largest eigenvalue $\lambda_{max}$ of the correlation matrix $\mathbf{X}$.  
%
We examine several commonly-available, pre-trained, production-quality DNNs by plotting $\hat{\alpha}$ versus the reported test accuracies.
This covers classes of DNN architectures including the VGG models, ResNet, DenseNet, etc. 
In nearly every class, and except for a few counterexamples, smaller $\hat{\alpha}$ corresponds to better average test accuracies, thereby providing a strong predictor of model quality.
%
We also show that this new complexity metric $\hat{\alpha}$ is approximately the average log of the Frobenius norm of the layer weight matrices, $\langle\log\Vert\mathbf{W}\Vert_{F}\rangle$ of Eqn.~(\ref{eqn:av_log_norm}), when accounting for finite-size effects.
This provides an interesting connection between the Statistical Physics approach to learning (from Martin and Mahoney~\cite{MM17_TR,MM18_TR}, that we extend here) and methods such as that of Liao et al.~\cite{LMBx18_TR}, who use norm-based capacity control metrics to bound worst-case generalization error.

We should mention two higher-level comments.
%
First, 
one of the main insights of our approach is to highlight the importance of what might seem to be a technical issue to ignore, but which in our experience is \emph{extremely} important: the scaling or normalization for weight matrices used in the DNN, and how this relates to the difference between finite-size versus asymptotic effects.
This scaling issue has been highlighted perhaps most recently by Bartlett et al.~\cite{BFT17_TR} and Liao et al.~\cite{LMBx18_TR}.
The latter were interested in showing that classical generalization bounds can be tight---when normalization is performed appropriately.
Our approach complements theirs; but, to our knowledge, our approach is the first to highlight the connection with finite-size effects.
%
Second, 
%it is worth emphasizing that 
we are taking a 
very %somewhat 
non-standard approach (at least for the DNN and ML communities) to address our main question.
We didn't train/retrain lots and lots of (typically rather small) models, analyzing training/test curves, trying to glean from them bits of insight that might then extrapolate to more realistic models.
Instead, we took advantage of the fact that there already exist many (typically rather large) publicly-available pre-trained models, and we analyzed the properties of these models.
That is, we viewed these publicly-available pre-trained models as artifacts of the world that achieve state-of-the-art performance in computer vision, natural language processing, and related applications; and we attempted to understand why.
To do so, we analyzed the empirical (spectral) properties of these models; 
%from this, we formed a hypothesis as to why they perform well; 
and we then extracted data-dependent metrics to predict their generalization performance on production-quality models.
Given well-known challenges associated with training, 
%and given our results here as well as other recent results~\cite{MM18_TR},
we suggest that this methodology be applied more generally.

Finally, one interesting aspect of our approach is that we can apply these complexity metrics \emph{across related DNN architectures}. 
This is in contrast to the standard practice in ML.
The equivalent notion would be to compare margins across SVMs, applied to the same data, but with different kernels. 
One loose interpretation is that a set of related of DNN models (i.e., VGG11, VGG13, etc.) is analogous to a single, very complicated kernel, and that the hierarchy of architectures is analogous to the hierarchy of hypothesis spaces in more traditional VC theory.
%\charlesX{more here ?  like this ?}
%\michael{Let's discuss this, to see what we can squeeze, given what is now popular.}   

We expect our result will have applications in the fine-tuning of DNN hyperparameters as well as related challenges.
Moreover, because we do not need to peek at the test data, our approach may prevent information from leaking from the test set into the model, thereby helping to prevent overtraining and making fined-tuned DNNs more robust.
Finally, our work also leads to a much harder theoretical question: is it possible to characterize properties of realistic DNNs to determine whether a DNN is overtrained---without peeking at the test data?  


