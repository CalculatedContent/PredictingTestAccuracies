
\begin{abstract}
\noindent
Given two or more Deep Neural Networks (DNNs) with the same or similar architectures, and trained on the same dataset, but trained with different solvers, parameters, hyper-parameters, regularization, etc., can we predict which DNN will have the best test accuracy, and can we do so without peeking at the test data?   
In this paper, we show how to use a new Theory of Heavy-Tailed Self-Regularization (HT-SR) for modern DNNs to answer this question. 
Based on HT-SR Theory, we develop a Universal capacity control metric that is a weighted average of the fitted layer power law exponents, where the weights depend on the log of the spectral norm of the correlations between layer weight matrices.
Rather than considering small toy NNs, we examine over 50 different, large-scale pre-trained DNNs, ranging over 15 different architectures, trained on ImagetNet, each of which has been reported to have different test accuracies.
We show that, across each architecture (VGG16/VGG19, 
%% WE DONT DO INCEPTION HERE %% InceptionV3/V4, 
ResNet10/ResNet101,
etc.), the reported test accuracies for each DNN are very well-correlated with this metric.
Moreover, we show this metric can be approximated by the average of the log of the Frobenius norm of the layer weight matrices.
Our approach requires no changes to the underlying DNN, it does not require us to train a model (although it could be used to monitor training), and it does not even require access to the ImageNet data.
\end{abstract}

