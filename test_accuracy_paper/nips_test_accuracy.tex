\documentclass{article}

% if you need to pass options to natbib, use, e.g.:
%     \PassOptionsToPackage{numbers, compress}{natbib}
% before loading neurips_2019

% ready for submission
 \usepackage{neurips_2019}

% to compile a preprint version, e.g., for submission to arXiv, add add the
% [preprint] option:
%     \usepackage[preprint]{neurips_2019}

% to compile a camera-ready version, add the [final] option, e.g.:
%%     \usepackage[final]{neurips_2019}

% to avoid loading the natbib package, add option nonatbib:
%     \usepackage[nonatbib]{neurips_2019}

%-----------------------------------------------------------------------

\usepackage{amsmath}
\usepackage{graphicx}
\usepackage{subfigure}
\usepackage{color}
\usepackage{xcolor}
\usepackage{amssymb}

\newcommand{\fix}[1]{\textcolor{red}{#1}}
\newcommand{\comment}[1]{\textcolor{blue}{#1}}
\newcommand{\awk}[1]{\textcolor{darkgreen}{#1}}

\newcommand{\argmin}{\text{argmin}}
\newcommand{\Probab}[1]{\mbox{}{\bf{Pr}}\left[#1\right]}
\newcommand{\Expect}[1]{\mbox{}{\bf{E}}\left[#1\right]}
\newcommand{\ExpectBracket}[1]{\mbox{}\langle#1\rangle}

% Here are two macros for comments.
\newcommand {\nred}[1]{{\color{red}\sf{[#1]}}}
\newcommand {\ngreen}[1]{{\color{darkgreen}\sf{[#1]}}}
\newcommand {\michael}[1]{{\color{red}\sf{[michael: #1]}}}
\newcommand {\charles}[1]{{\color{blue}\sf{[charles: #1]}}}
\newcommand {\charlesX}[1]{{\color{violet}\sf{[charlesG: #1]}}}

\usepackage[normalem]{ulem}

%-----------------------------------------------------------------------

\usepackage[utf8]{inputenc} % allow utf-8 input
\usepackage[T1]{fontenc}    % use 8-bit T1 fonts
\usepackage{hyperref}       % hyperlinks
\usepackage{url}            % simple URL typesetting
\usepackage{booktabs}       % professional-quality tables
\usepackage{amsfonts}       % blackboard math symbols
\usepackage{nicefrac}       % compact symbols for 1/2, etc.
\usepackage{microtype}      % microtypography

\title{Heavy-Tailed Universality Predicts Trends in Test Accuracies for Very Large Pre-Trained Deep Neural Networks}

% The \author macro works with any number of authors. There are two commands
% used to separate the names and addresses of multiple authors: \And and \AND.
%
% Using \And between authors leaves it to LaTeX to determine where to break the
% lines. Using \AND forces a line break at that point. So, if LaTeX puts 3 of 4
% authors names on the first line, and the last on the second line, try using
% \AND instead of \And before the third author name.

\author{%
  Charles H. Martin \\
  Calculation Consulting \\
  8 Locksley Ave, 6B \\
  San Francisco, CA 94122 \\
  \texttt{charles@CalculationConsulting.com} \\
  \And
  Michael W. Mahoney \\
  ICSI and Department of Statistics \\
  University of California at Berkeley \\
  Berkeley, CA 94720 \\
  \texttt{mmahoney@stat.berkeley.edu} \\
}

\begin{document}

\maketitle

\input{nips_test_accuracy_abstract}
\input{nips_test_accuracy_intro}

\vspace{-3mm}

\section{Brief Overview of Heavy-Tailed Self-Regularization}
\label{sxn:theory-review_abridged}

\vspace{-2mm}

Here, we briefly review Martin and Mahoney's Theory of Heavy-Tailed Self-Regularization (HT-SR)~\cite{MM18_TR,MM19_HTSR_ICML}.
See Appendix~\ref{sxn:theory-review} for more details.

Write the Energy Landscape (or optimization function) for a typical DNN with $L$ layers, with activation functions $h_{l}(\cdot)$, and with $N\times M$ weight matrices $\mathbf{W}_{l}$ and biases $\mathbf{b}_{l}$, as:
\begin{equation*}
%PRESQUISH% E_{DNN}=h_{L}(\mathbf{W}_{L}\times h_{L-1}(\mathbf{W}_{L-1}\times h_{L-2}(\cdots)+\mathbf{b}_{L-1})+\mathbf{b}_{L})  .
E_{DNN} \hspace{-1mm} = \hspace{-1mm} h_{L}(\mathbf{W}_{L}\cdot h_{L-1}(\mathbf{W}_{L-1}\cdot h_{L-2}(\cdots)+\mathbf{b}_{L-1})+\mathbf{b}_{L})  .
%\label{eqn:dnn_energy}
\end{equation*}
%WLOG,
Typically, this model would be trained on some labeled data $\{d_{i},y_{i}\}\in\mathcal{D}$, using Backprop, by minimizing the loss $\mathcal{L}$.
For simplicity, we do not indicate the structural details of the layers (e.g., Dense or not, Convolutions or not, Residual/Skip Connections, etc.). 
%Each layer is defined by, e.g., one or more layer 2D weight matrices $\mathbf{W}_{l}$, and/or the 2D feature maps $\mathbf{W}_{l,i}$ extracted from 2D Convolutional (Conv2D) layers.

In the HT-SR Theory, we analyze the eigenvalue spectrum (the ESD) of the associated correlation matrices~\cite{MM18_TR,MM19_HTSR_ICML}.
From this, we can characterize the amount and form of correlation, and therefore implicit self-regularizartion, present in the DNN's weight matrices.
For each layer weight matrix, of size $N \times M$, construct the associated $M\times M$ (uncentered) correlation matrix $\mathbf{X}$. 
Dropping the $L$ and $l,i$ indices, we have
$
\mathbf{X} = \frac{1}{N}\mathbf{W}^{T}\mathbf{W}.
$
If we compute the eigenvalue spectrum of $\mathbf{X}$, i.e., $\lambda_i$ such that
$  % $$
\mathbf{X}\mathbf{v}_{i}=\lambda_{i}\mathbf{v}_{i} , 
$  % $$
then the ESD of eigenvalues, $\rho(\lambda)$, is just a histogram of the eigenvalues.
Using HT-SR Theory, we can characterize the \emph{correlations} in a weight matrix by examining its ESD, $\rho(\lambda)$.
It can be well-fit to a power law (PL) distribution, given as
$
\rho(\lambda)\sim\lambda^{-\alpha}  ,
$
which is (at least) valid within a bounded range of eigenvalues $\lambda\in[\lambda^{min},\lambda^{max}]$.  

In Statistical Physics, Universality 
arises in systems with very strong correlations, at or near a critical point or phase transition. 
It is characterized by measuring experimentally certain ``observables'' that display HT behavior, with common---or Universal---PL exponents. 
More importantly, it indicates that a specific Universal mechanism drives the underlying physical process, e.g., Self Organized Criticality, directed percolation, etc.~\cite{SornetteBook,BouchaudPotters03}. 
For this reason, we refer to the Universality observed in HT-SR, i.e., in the ESDs of (pre-trtained) DNN weight matrices, as \emph{Heavy-Tailed Mechanistic Universality~(HT-MU)}.

When we observe HT behavior in $\mathbf{W}$, or rather its correlation matrix $\mathbf{X}$, we use HT-RMT as a generative model. 
We say that we \emph{model} $\mathbf{W}$ \emph{as if} it is a random matrix, $\mathbf{W}^{rand}(\mu)$, drawn from a Universality class of HT-RMT (i.e., VHT, MHT, or WHT, as defined below). 
%
To characterize this HT-MU behavior, we use a HT variant of RMT and use HT random matrices to elucidate different Universality classes.
Let $\mathbf{W}(\mu)$ be an $N \times M$ random matrix with entries chosen i.i.d. from
$$
\Probab{ W_{i,j} } \sim \frac{W_{0}^{\mu}}{|W_{i,j}|^{1+\mu}}  ,
$$
where $W_{0}$ is the typical order of magnitude of $W_{i,j}$, and where $\mu>0$. 
There are at least 3 different Universality classes
of HT random matrices, defined by the range $\mu$ takes on:
\begin{itemize}
\item $0<\mu<2$: VHT: Universality class of Very Heavy-Tailed (or L\'evy) matrices;
\item $2<\mu<4$: MHT: Universality class of Moderately Heavy-Tailed (or Fat-Tailed) matrices;
\item $4<\mu$: WHT: Universality class of Weakly Heavy-Tailed matrices.
\end{itemize}

HT-RMT provides more than HT Universality classes.
It also provides simple relations between the empirical observables, e.g., the PL exponent $\alpha$ and the maximum eigenvalue $\lambda^{max}$ of each $\mathbf{W}$, with the parameter(s) $\mu$ of our generative theory, i.e, of~HT-RMT.   
As described in Appendix~\ref{sxn:theory-review}, \emph{due to Heavy Tailed Mechanistic Universality (HT-MU)}, we expect 
$$
\text{VHT\;\&\;MHT:}\;\;\;\lambda^{max}\sim N^{4/\mu-1}  
$$
(where, for simplicity, $Q=1$)  
to hold for matrices in these HT Universality classes (as evidenced by their ESD properties), e.g., DNN weight matrices $\mathbf{W}$ after training---\emph{even when the matrix is not itself a HT random matrix} and therefore not governed by RMT.
The $\alpha$ and $\lambda^{max}$ are empirically-measurable quantities---of real or synthetic matrices---while $\mu$ is a parameter of the HT-RMT model. 
We shall use these Universal HT finite-size relations to derive a simple capacity control metric for our HT-SR Theory, and relate this to the well known Product Norm capacity control metric.






\vspace{-3mm}

%\section{Using Heavy-Tailed Universality}
\section{Heavy-Tailed Mechanistic Universality and Capacity Control Metrics}
\label{sxn:theory-new}

\vspace{-2mm}

%\charlesX{\bf{Heavy-Tailed Self-Regularization and Capacity Control Metrics}}

%In this section, we will describe our proposed capacity control metric.
%
From prior work~\cite{MM18_TR,MM19_HTSR_ICML}, we expect that smaller PL exponents of the ESD imply more regularization and therefore better generalization. 
Since smaller norms of weight matrices often correspond to better capacity control~\cite{LMBx18_TR,SHNx17_TR,PLMx18_TR,BFT17_TR}, we would like to relate the empirical PL exponent $\alpha$ to the empirical Frobenius norm $\Vert\mathbf{W}\Vert_{F}$.
At least na\"{\i}vely, this is a challenge, since smaller PL exponents often correspond to larger matrix norms (and thus worse generalization!).
See Appendices~\ref{sxn:appendix-random-vs-real} and~\ref{sxn:appendix-universality}.
To resolve this apparent discrepancy, we will exploit HT-MU to propose a Universal DNN complexity metric.

\paragraph{Form of a Proposed Universal DNN Complexity Metric.} 

The PL exponent $\alpha$ is a complexity metric for a single DNN weight matrix, with smaller values corresponding to greater regularization~\cite{MM18_TR,MM19_HTSR_ICML}.
% The fitted PL  ... AWK
It describes how well that matrix encodes complex correlations in the training data.
Thus, a natural class of complexity or capacity metrics to consider for a DNN is to take a \emph{weighted average}%
\footnote{There are several reasons we don't want an unweighted average: 
an unweighted average behaves differently for HT random matrices than for well-trained DNN weight matrices, and so it would not be Universal; 
we want a metric that relates the $\alpha$ of HT-SR Theory with known capacity control metrics such as norms of weight matrices, and including weights permits this flexibility; 
we want weights to encode information that ``larger'' matrices are somehow more important;
and unweighted averages, while sometimes providing predictive quality, do not perform as reliably~well. 
See Appendices~\ref{sxn:appendix-random-vs-real} and~\ref{sxn:appendix-universality} for more details.
}
of the PL exponents, $\alpha_{l,i}$, for each layer weight matrix $\mathbf{W}_{l,i}$:
\begin{equation}
\hat{\alpha}:=\dfrac{1}{N_L}\sum_{l,i}b_{l,i}\alpha_{l,i}  .
\label{eqn:alpha_hat_generic}
\end{equation}
Here, the smaller $\hat{\alpha}$, the better we expect the DNN to represent training data, and (presumably) the better the DNN will generalize.  % to new data.
An open question is: what are good weights~$b_{l,i}$?

As we now show, we can extract the weighted average $\hat{\alpha}$ directly from the more familiar Product Norm, by exploiting both HT Universality, and its finite-size effects, arising
in DNN weight matrices.


%%%\paragraph{THEOREM:} \emph{The data dependent VC-like complexity of a Deep Neural Network can be expressed a weighted average the of power law exponents describing the empirical spectral density of layer weight matrices}
%%
%%%\charles{\paragraph{PROOF:...}}


\paragraph{Product Norm Measures of Complexity.} 

%% XXX. I PUT THIS MOSTLY IN THE INTRO.
%% \charlesX{NEED TO CLARIFY 
%% Worst case Bounds vs Average Case for complexity metrics, and REVIEW MORE OF HIDary's work, either here and/or in the Intro}
%% \michael{Their method works well on toy data for worst-case, and to get it to work they need to modify the loss function in worse-case, but if we consider average case then we can apply it to large realistic DNNs---put these comments here or in intro.}

It has been suggested that the complexity, $\mathcal{C}$, of a DNN can be characterized by the product of the norms of layer weight matrices,
$$
\mathcal{C}\sim\Vert\mathbf{W}_{1}\Vert\times\Vert\mathbf{W}_{2}\Vert\cdots\Vert\mathbf{W}_{L}\Vert ,
$$
where $\Vert\mathbf{W}\Vert$ is, e.g., the Frobenius norm~\cite{LMBx18_TR, SHNx17_TR,PLMx18_TR}.
(Here, we can use either $\Vert\mathbf{W}\Vert$ or $\Vert\mathbf{W}\Vert^{2}$, and one can view $\mathcal{C}$ as akin to a data-dependent VC complexity.)
To that end, we consider a log~complexity
\begin{eqnarray*}
\log\mathcal{C} &\sim& \log\bigg[\Vert\mathbf{W}_{1}\Vert\times\Vert\mathbf{W}_{2}\Vert\cdots\Vert\mathbf{W}_{L}\Vert\bigg]  \\
                &\sim& \bigg[\log\Vert\mathbf{W}_{1}\Vert+\log\Vert\mathbf{W}_{2}\Vert\cdots\log\Vert\mathbf{W}_{L}\Vert\bigg]  ,
\end{eqnarray*}
and we define the average log norm of weight matrices (where $N_{L}$ is the number of layers)~as
\begin{equation}
\langle\log\Vert\mathbf{W}\Vert\rangle=\dfrac{1}{N_{L}}\sum_{l}\log\Vert\mathbf{W}_{l}\Vert  .
\label{eqn:av_log_norm}
\end{equation}

%% \michael{Ques: is the notation for layers or convolutions or what, be consistent with Eqn.~(\ref{eqn:alpha_hat_generic}).}
%% \charlesX{Need more references to Hidary's work}
%% MM: A BIT HERE, BUT MOST IS IN INTRO AND CONCLUSION.


\paragraph{A Universal, Linear, PL--Norm Relation.} 

We propose a simple linear relation between the (squared) Frobenius norm $\Vert\mathbf{W}\Vert^{2}_{F}$ of $\mathbf{W}$, the PL exponent $\alpha$, and the maximum eigenvalue $\lambda^{max}$ of $\mathbf{X}$ (i.e., the spectral norm $\Vert\mathbf{X}\Vert_{2}=\frac{1}{N}\Vert\mathbf{W}\Vert^{2}_{2}$):  
\begin{equation}
\textbf{PL--Norm Relation:} \quad \alpha\log\lambda^{max}\approx\log\Vert\mathbf{W}\Vert^{2}_{F}  .
\label{eqn:basic_relation}
\end{equation}
To our knowledge, this is the first time this PL--Norm relation has been noted in the literature (although prior work has considered norm bounds for HT data~\cite{MN09_TR}).
A few comments on Eqn.~(\ref{eqn:basic_relation}).
First, it provides a connection between the PL parameter $\alpha$ of HT-SR Theory and the weight norm $\Vert\mathbf{W}\Vert^{2}_{F}$ of more traditional statistical learning theory.
Second, it has the form of the well-known Hausdorff dimension~\cite{Sch07}.
Third, it shows that PL exponents can alternatively be interpreted (up to the $\frac{1}{N}$ scaling) as the Stable Rank in Log-Units:
$$
\mbox{Log-Units Stable Rank:} 
\quad
\mathcal{R}^{log}_{s}:=\dfrac{\log\Vert\mathbf{W}\Vert^{2}_{F}}{\log\lambda^{max}}  \approx \alpha  .
$$
Our justification for proposing Eqn.~(\ref{eqn:basic_relation}) is three-fold.
%%\charles{GOOD}
\begin{enumerate}
\item
\label{enum:first}
We derive Eqn.~(\ref{eqn:basic_relation}) in the special case of very small PL exponent, $\alpha \rightarrow 1$ ($\mu\rightarrow 0$), for an $N\times M$ 
%random 
matrix $\mathbf{W}^{rand}(\mu)$ (with $N=M$, or $Q=1$).
See Appendix~\ref{sxn:appendix-derivation-pl-norm-relation}.
\item
\label{enum:second}
For finite-size random matrices $\mathbf{W}^{rand}(\mu)$, the MHT Universality class, $\mu\in(2,4)$, behaves \emph{like} the VHT Universality class, $\mu\in(1,2)$.
Because of this similarity, we show we can extend Eqn.~(\ref{eqn:basic_relation}), approximately, to larger PL exponents.
For $N\sim\mathcal{O}(100-1000)$, $\alpha\log\lambda^{max}$ increases nearly linearly with $\log\Vert\mathbf{W}^{rand}(\mu)\Vert^{2}_{F}$ as $\mu$ increases.
For larger $N$, the relation saturates for large $\mu$. 
See Appendix~\ref{sxn:appendix-finite-size}.
\item
\label{enum:third}
\emph{As evidence of HT-MU}, we observe empirically that Eqn.~(\ref{eqn:basic_relation}) also applies, approximately, to the real DNN weight matrices $\mathbf{W}$. 
We see that $\alpha\log\lambda^{max}$ is positively correlated with $\log\Vert\mathbf{W}\Vert^{2}_{F}$ as $\alpha$ increases, and even shows similar saturation effects at large $\alpha$.
See Appendix~\ref{sxn:appendix-random-vs-real}.
\end{enumerate}

%%SPACE MOVED TO APPENDIX.%% We will discuss each of these in more detail below.

Finally, based on Eqn.~(\ref{eqn:basic_relation}), we choose the weights in Eqn.~(\ref{eqn:alpha_hat_generic}) to be the log of the corresponding maximum eigenvalues of $\mathbf{X}$.
That is, for a given $l,i$, we have the weights in Eqn.~(\ref{eqn:alpha_hat_generic})~as
$$
b_{l,i} = \lambda_{l,i}^{max}  .
$$
Then, we define the complexity metrics for Linear and Convolutional Layers as follows:
%% $$
%% \text{Linear Layer:}\;\;\log\Vert\mathbf{W}_{L}\Vert^{2}_{F}\rightarrow\log\lambda^{max}_{L}\alpha_{L}  .
%% $$
%% \michael{Need to be consistent with superscripts and subscripts, on $\lambda$, in this par and elsewhere.}
%% $$
%% \text{Conv2D Layer:}\;\;\log\Vert\mathbf{W}_{L}\Vert^{2}_{F}\rightarrow \sum_{i=1}^{n_{L}}\log\lambda^{max}_{i,L}\alpha_{L,i}  .
%% $$
%% I CHANGED THIS EVERyWHERE %% \nred{Hard to read max..make superscript ?}
\begin{eqnarray*}
\text{Linear Layer:} & & \log\Vert\mathbf{W}_{l}\Vert^{2}_{F} 
%\quad 
\rightarrow 
%\quad 
\alpha_{l}\log\lambda_{l}^{max}  \\
\text{Conv2D Layer:} & & \log\Vert\mathbf{W}_{l}\Vert^{2}_{F} 
%\quad 
\rightarrow 
%\quad 
\sum_{i=1}^{n_{l}}\alpha_{l,i} \log\lambda_{l,i}^{max} , 
\end{eqnarray*}
where, for Conv2D Layers, we relate the ``norm'' of the 4-index Tensor $\mathbf{W}_{l}$ to the sum of the $n_{l}=c\times d$ terms for each feature map.
%% So, in the expression for the Product Norm for $\log\mathcal{C}$, we can replace each $\log\Vert\mathbf{W}_{L}\Vert$ term for layer $L$ with these above expressions, and take the average over all $N_{\alpha}$  matrices.  
This lets us compare the Product Norm to the weighted average of PL exponents as follows:
\begin{equation}
2\log\mathcal{C}=\langle\log\Vert\mathbf{W}\Vert^{2}_{F}\rangle 
%\quad 
\rightarrow 
%\quad 
\hat{\alpha} := \dfrac{1}{N_{L}}\sum_{i,l}\alpha_{i,l}\log \lambda_{l,i}^{max}  .
\label{eqn:alpha_hat_specific}
\end{equation}
%% \michael{We are using subscripts in a slightly confusing way.}
%%
%% 
%% This expression resembles the more familiar Product Norm, but it accounts for finite-size effects that the Product Norm relation over-estimates.
%% \michael{Ques: is that true.} \charlesX{probably not}
%% 
%% We will see that our approach improves on the loose bound provided by the Product Norm, giving a more accurate expression for predicting trends in the average case test accuracy for real-world production-quality DNNs.
%%
Given these connections, in Section~\ref{sxn:emp}, we will use $\hat{\alpha}$ to analyze numerous pre-trained DNNs.


%% \charlesX{THE NEXT 2 PARAGRAPHICAL SECTIONS EXPAND ON THE ABOVE POINTS. THE FIRST DISCUSSES THE FACT THAT A RANDOM HT MATRIX HAS A DIVERGING NORM, WHEREAS THE CORRELATED MATRICES HAVE SMALLER NORMS.  THIS IS EXPECTED FROM THEORY ALSO (CITE THE CHICAGO GUYS).  





\vspace{-3mm}

\section{Empirical Results on Pre-trained DNNs}
\label{sxn:emp}

\vspace{-2mm}

Here, we summarize our empirical results for the VGG and ResNet series of models.
See Appendix~\ref{sxn:appendix-addl-empirical} for additional empirical results on other pre-trained DNN models.

We only consider Linear and Conv2D layers because we will only examine series of commonly available, open source, pre-trained DNNs with these kinds of layers. 
All models have been trained on ImageNet, and reported test accuracies are widely available. 
Throughout, we use the Test Accuracies for the Top1 errors (where Accuracy = 100 - Top1 error).
We see similar results for the Top5 errors.
We emphasize that, \emph{for our analysis, we do not need to retrain these models---and we do not even need the test data!}

\paragraph{VGG and VGG\_BN Models.}

\begin{figure*}[t] %[!htb]
   \centering
   \subfigure[log Frobenius norm $\langle\log\Vert\mathbf{W}\Vert_{F}\rangle$]{
      \includegraphics[scale=0.25]{img/vgg-lognorms.png}
      \label{fig:vgg_lognorms}
   }
   \subfigure[weighted average PL exponent $\hat{\alpha}$]{
      \includegraphics[scale=0.25]{img/vgg-w_alphas.png}
      \label{fig:vgg_alphahat}
   }
   \caption{%
      Pre-trained VGG and VGG\_BN Architectures and DNNs.  
      Top 1 Test Accuracy versus
      average log Frobenius norm $\langle\log\Vert\mathbf{W}\Vert_{F}\rangle$ (in (\ref{fig:vgg_lognorms}))
      or
      Universal, weighted average PL exponent $\hat{\alpha}$ (in (\ref{fig:vgg_alphahat}))
      for
      VGG11 vs VGG11\_BN ({\color{blue}{blue}}),
      VGG13 vs VGG13\_BN ({\color{orange}{orange}}),
      VGG16 vs VGG16\_BN ({\color{green}{green}}),  and
      VGG19 vs VGG19\_BN ({\color{red}{red}}). 
      We plot plain the VGG models with circles and the VGG\_BN models with~squares.
   }
   \label{fig:vgg}
\end{figure*}


%% COMBINED WITH ABOVE %% \begin{figure}[!htb]
%% COMBINED WITH ABOVE %%  \centering
%% COMBINED WITH ABOVE %%    \includegraphics[scale=0.40]{img/vgg-w_alphas.png}
%% COMBINED WITH ABOVE %%    \caption{
%% COMBINED WITH ABOVE %% Pre-trained VGG and VGG BN Architectures and DNNs.  Test Accuracy and weighted average $\hat{\alpha}$ for
%% COMBINED WITH ABOVE %%  VGG11 vs VGG11\_BN ({\color{blue}{blue}}),
%% COMBINED WITH ABOVE %% VGG13 vs VGG13\_BN ({\color{orange}{orange}}),
%% COMBINED WITH ABOVE %% VGG16 vs VGG16\_BN ({\color{green}{green}}),  and
%% COMBINED WITH ABOVE %% VGG19 vs VGG19\_BN ({\color{red}{red}}). 
%% COMBINED WITH ABOVE %% }
%% COMBINED WITH ABOVE %%   \label{fig:vgg_alphahat}
%% COMBINED WITH ABOVE %% \end{figure}



\begin{table}[t]
\small
\begin{center}
\begin{tabular}{|p{0.75in}|c|c|c|c|c|c|c|}
\hline
Model & Top1 Accuracy & $\hat{\alpha}$ \\
\hline
VGG11 & 68.97 & 1.84 \\
VGG11\_BN & 70.45 & 1.60 \\
\hline
VGG13 & 69.66 & 1.65 \\
VGG13\_BN & 71.51 & 1.36 \\
\hline
VGG16 & 71.64 & 1.41 \\
VGG16\_BN & 73.52 & 1.08 \\
\hline
VGG19 & 72.08 & 1.16 \\
VGG19\_BN & 74.27 & 0.81 \\
\hline
\end{tabular}
\end{center}
\caption{%
         Results for VGG Architecture.   The Top1 Accuracy is defined
as the $100.0$ minus the Top1 reported error.
         }
\label{table:models_VGG}
\end{table}


We first look at the VGG class of models, comparing the log norm and the Universal $\hat{\alpha}$ metrics.
See Figure~\ref{fig:vgg} and Table~\ref{table:models_VGG} for a summary of the results.
Figures~\ref{fig:vgg_lognorms} and~\ref{fig:vgg_alphahat} show both the average log Frobenius norm, $\langle\log\Vert\mathbf{W}\Vert_{F}\rangle$ of Eqn.~(\ref{eqn:av_log_norm}), and the weighted average PL exponent, $\hat{\alpha}$ of Eqn.~(\ref{eqn:alpha_hat_specific}), as a function of the reported (Top1) test accuracy for the series of pre-trained VGG models, as available in the pyTorch package.%
\footnote{\url{https://pytorch.org/}}
These models include VGG11, VGG13, VGG16, and VGG19, as well as their more accurate counterparts with Batch Normalization, VGG11\_BN, VGG13\_BN, VGG16\_BN and VGG19\_BN. 
%See Figures~\ref{fig:vgg_lognorms} and~\ref{fig:vgg_alphahat} as well as Table~\ref{table:models_VGG} for details.
Table~\ref{table:models_VGG} provides additional details.

%Figure~\ref{fig:vgg_lognorms} shows the average log Frobenius norm results, which are quite good; and 
%Figure \ref{fig:vgg_alphahat} shows the weighted average PL exponent results, which yield slight improvements due to the method we introduce.
Across the entire series of architectures, 
reported test accuracies increase linearly as each metric, 
%%the average log Frobenius norm 
$\langle\log\Vert\mathbf{W}\Vert_{F}\rangle$
%%, Eqn.~(\ref{eqn:av_log_norm}), 
and 
%%the average weighted power law exponent 
$\hat{\alpha}$,
%%, Eqn.~(\ref{eqn}).
decreases.
Moreover, whereas the log norm relation has 2 outliers, VGG13 and VGG13\_BN, the Universal $\hat{\alpha}$ metric shows a near perfect linear relation across the entire VGG~series.

\paragraph{ResNet Models.}

\begin{figure*}[!htb]
   \centering
   \subfigure[log Frobenius norm $\langle\log\Vert\mathbf{W}\Vert_{F}\rangle$]{
      \includegraphics[scale=0.21]{img/ResNet_top1-lognorms.png}
      \label{fig:resnet_lognorms}
   }
   \subfigure[weighted average PL exponent $\hat{\alpha}$]{
      \includegraphics[scale=0.21]{img/ResNet-w_alphas.png}
      \label{fig:resnet_alphahat}
   }
   \caption{
      Pre-trained
      ResNet Architectures and DNNs.  
      Top 1 Test Accuracy versus
      average log Frobenius norm $\langle\log\Vert\mathbf{W}\Vert_{F}\rangle$ (in (\ref{fig:resnet_lognorms}))
      or
      Universal, weighted average PL exponent $\hat{\alpha}$ (in (\ref{fig:resnet_alphahat})).
           }
   \label{fig:resnet}
\end{figure*}

\begin{table}[t] %[!htb]
\small
\begin{center}
\begin{tabular}{|p{0.75in}|c|c|c|c|c|c|c|}
\hline
Architecture 
 & Model
 & Top 1 Accuracy & $\hat{\alpha}$ \\
\hline
ResNet (small)   & resnet10 & 62.54 & 1.94 \\
 & resnet12 & 63.82 & 0.74 \\
 & resnet14 & 66.83 & 1.70 \\
 & resnet16 & 69.10 & 1.49 \\
 \hline
 ResNet18  & resnet18\_wd4 & 50.50 & 1.83 \\
 & resnet18\_wd2 & 62.96 & 1.82 \\
 & resnet18\_w3d4 & 66.39 & 0.28 \\
 & resnet18 & 70.48 & 1.09 \\
 \hline
ResNet34 & resnet34 & 74.34 & -0.42 \\
\hline
ResNet50  & resnet50 & 76.21 & 0.13 \\
 & resnet50b & 76.95 & 0.09 \\
 \hline
ResNet101 & resnet101 & 78.10 & -0.67 \\
 & resnet101b & 78.55 & -0.92 \\
\hline
ResNet152 & resnet152 & 78.74 & -1.11 \\
 & resnet152b & 79.26 & -1.74 \\
\hline
\end{tabular}
\end{center}
\caption{Results for ResNet Architectures and DNN Models.  The Top1 Accuracy is defined
as the $100.0$ minus the Top1 reported error.  Some $\hat{\alpha}<0$ because the of how
the ResNet weight matrices are internally scale and normalized, which makes the maximum eigenvalue
less then one, $\lambda^{max}<1$.
        }
\label{table:models_resnet}
\end{table}


\vspace{-1mm}

We next look at the ResNet class of models. 
See Figure~\ref{fig:resnet} and Table~\ref{table:models_resnet} for a summary of the results.
Here, we consider a set of 15 different pre-trained ResNet models, of varying sizes and accuracies, ranging from the small ResNet10 up to the largest ResNet152 models, as provided by the OSMR sandbox,%
\footnote{\url{https://github.com/osmr/imgclsmob}}
developed for training large-scale image classification networks for embedded systems.
Again, we compare the reported (Top1) test accuracy versus the average log norm $\langle\log\Vert\mathbf{W}\Vert_{F}\rangle$ and the Universal $\hat{\alpha}$ metrics. 

As with the VGG series, both metrics monotonically decrease as the test accuracies decrease for the ResNet series, and both metrics have a few large outliers off the main line relation. 
See Figures~\ref{fig:resnet_lognorms} and~\ref{fig:resnet_alphahat}.
In particular, the log norm metric has several notable outliers, including resnet18\_wd2, resnet18\_wd3\_d4, resnet34, and resnet10. 
The $\hat{\alpha}$ metric shows a slightly better relation, with resnet18\_wd2 more in line, and the other 3 outliers a little less off the main line of correlation. 
The Universal $\hat{\alpha}$ metric is as good or slightly better than the average log norm metric for the Resnet series of models. 

We see similar results for our Universal PL capacity control metric $\hat{\alpha}$ across a wide range of other pre-trained DNN models, described in Appendix~\ref{sxn:appendix-addl-empirical}.
In nearly all cases, the metric $\hat{\alpha}$ correlates well with the reported test accuracies, with only a three DNN architectures as exceptions. 
Overall the $\hat{\alpha}$ metric systematically correlates well with the generalization accuracy of a wide class of pre-trained DNN architectures---which is rather remarkable.


\vspace{-2mm}
\section{Discussion and Conclusion}
\label{sxn:discussion}
\vspace{-1mm}

We have presented an \emph{unsupervised} capacity control metric which predicts trends in test accuracies of a trained DNN---without peeking at the test data. 
In the interests of space, see Appendix~\ref{sxn:appendix-addl-discussion} for additional discussion.
We conclude by observing simply that 
%We expect our result will have applications in the fine-tuning of DNN hyperparameters as well as related challenges.
%Moreover, because we do not need to peek at the test data, our approach may prevent information from leaking from the test set into the model, thereby helping to prevent overtraining and making fined-tuned DNNs more robust.
%Finally, 
our work also leads to a much harder theoretical question: is it possible to characterize properties of realistic DNNs to determine whether a DNN is overtrained---without peeking at the test data?  




\bibliographystyle{unsrt}
%\bibliographystyle{plain}

{\small
%\bibliography{gen_gap}
%\bibliography{dnns}
\bibliography{dnns,gen_gap}
}

\appendix


\newpage


%\paragraph{The PL--Norm Relation: Deriving a Special Case of Eqn.~(\ref{eqn:basic_relation}).}
\section{The PL--Norm Relation: Deriving a Special Case of Eqn.~(\ref{eqn:basic_relation})}
\label{sxn:appendix-derivation-pl-norm-relation}

Here, we derive Eqn.~(\ref{eqn:basic_relation}) in the special case of very small PL exponent, 
as 
%%$\alpha \rightarrow 1$, 
$\mu \rightarrow 0$, 
for an $N \times M$ random matrix $\mathbf{W}$, {with $M=N, Q=1$, and with elements drawn from Eqn.~(\ref{eqn:ht_dstbn}).%
\footnote{We derive Eqn.~(\ref{eqn:basic_relation}) at what is sometimes pejoratively known as ``at a physics level of rigor.''  That is fine, as our justification ultimately lies in our empirical results.  Recall our goal: to derive a very simple expression relating fitted PL exponents and Frobenius norms that is usable by practical engineers working with state-of-the-art models, i.e., not simply small toy models.  There is very little ``rigorous'' work on HT-RMT, less still on understanding finite-sized effects of HT Universality.  Hopefully, our results will lead to more work along these lines.  }
We seek a relation good in the region $\mu\in[0,2]$, and we will extend the $\mu\sim 0$ results to this full region.
That is, we establish this as an asymptotic relation for the VHT Universality class for very small~exponents.

To start, recall that 
$$ 
\Vert \mathbf{W}\Vert_{F}^{2}=\mbox{Trace}[\mathbf{W}^{T}\mathbf{W}]=N\;\mbox{Trace}[\mathbf{X}]  .
$$
Since, $\mu \gtrsim 0$, 
the eigenvalue spectrum is dominated by a single large eigenvalue, it follows that
$$
\Vert \mathbf{W}\Vert_{F}^{2}\approx N\lambda^{max}  , 
$$
where $\lambda^{max}$ is the largest eigenvalue of the matrix $\mathbf{X}$ (with the $1/N$ normalization).
Taking the log of both sides of this expression and expanding leads to
%\begin{eqnarray*}
%\log\Vert \mathbf{W}\Vert_{F}^{2} 
%   &\approx& \log \left( N\lambda^{max} \right) \\
%   &=&       \log N+\log\lambda^{max}  .
%\end{eqnarray*}
\begin{eqnarray*}
\log\Vert \mathbf{W}\Vert_{F}^{2} 
   \approx \log \left( N\lambda^{max} \right) 
   =       \log N+\log\lambda^{max}  .
\end{eqnarray*}
Rearranging, we get that 
$$
\dfrac{\log\Vert \mathbf{W}\Vert_{F}^{2}}{\log\lambda^{max}}\approx \dfrac{\log N}{\log\lambda^{max}}+1  .
$$
Thus, for a parameter $\alpha$ satisfying Eqn.~(\ref{eqn:basic_relation}), we have that 
$$
\alpha\approx \dfrac{\log N}{\log\lambda^{max}}+1  .
$$
%%%
%%%$$
%%%\alpha-1\approx \dfrac{\log N}{\log\lambda^{max}}  .
%%%$$
%%%
Recall that the relation between $\alpha$ and $\mu$ for the VHT Universality class is given in Eqn.~(\ref{eqn:alpha_mu_vht}) as
$ %$$
\alpha=\frac{1}{2}\mu+1  .
$ %$$
Thus, to establish our result, we need to show that
$$
\dfrac{\log N}{\log\lambda^{max}}\approx\dfrac{1}{2}\mu  .
$$
To do this, we use the relation of Eqn.~(\ref{eqn:scaling_of_lambda_max}) for the tail statistic, i.e., that 
$ %$$
\lambda^{max}\approx N^{4/\mu-1}  .
$ %$$
Taking the log of both sides gives
$$
\log\lambda^{max}\approx\log N^{4/\mu-1}=(4/\mu-1)\log N  ,
$$
from which it follows that
$$
\dfrac{\log N}{\log\lambda^{max}}\approx\dfrac{\log N}{(4/\mu-1)\log N}=\dfrac{1}{4/\mu-1}   .
$$
Finally, we can form the Taylor Series for $\dfrac{1}{4/\mu-1}$ around, e.g., $\mu=1.15\approx 1$, which gives 
$$
\dfrac{1}{4/\mu-1}\bigg\rvert_{\mu=1.15}\approx\dfrac{1}{2}\mu-\dfrac{1}{6}+\cdots\approx\dfrac{1}{2}\mu  .
$$
This relation is depicted in Figure~\ref{fig:taylor-series}.
This establishes the approximate---and rather surprising---linear relation we want for $\mu\in[0,2]$ for
the VHT Universality class of HT-RMT.

\begin{figure}[!htb]
   \centering
   \includegraphics[scale=0.30]{img/taylor-series.png} 
  \caption{%
           Taylor series expansion for $\frac{1}{4/\mu-1}$ at $\mu=1.15\approx1$.
          }
  \label{fig:taylor-series}
\end{figure}

%\paragraph{The PL--Norm Relation: Finite-Size Effects.}
\section{The PL--Norm Relation: Finite-Size Effects.}
\label{sxn:appendix-finite-size}


Here, we consider finite-size effects in Eqn.~(\ref{eqn:basic_relation}), both within and across HT Universality classes, i.e., for both VHT and MHT matrices.
See Figure~\ref{fig:randW}, which  displays $\frac{\log\Vert\mathbf{W}\Vert^{2}_{F}}{\log\lambda^{max}}$ as a function of the fitted PL exponent $\alpha$, with varying sizes $N$ (with aspect ratio $Q=1$).
Recall that $\alpha \approx \frac{1}{2}\mu+1$ for VHT random matrices (Eqn.~(\ref{eqn:alpha_mu_vht})), while $\alpha = a\mu+b$ for MHT random matrices (Eqn.~(\ref{eqn:alpha_mu_mht})), where $a,b$ strongly depend on $N$ and $M$.
Thus, $\mu\in(0,2)$ for VHT matrices corresponds to $\alpha\in(1,2)$, while $\alpha \approx(2,5)$ for MHT matrices.
%% $$
%% \dfrac{\log\Vert\mathbf{W}\Vert^{2}_{F}}{\log\lambda^{max}}\;\;vs.\;\;(\alpha)  .
%% $$

The numerical results in Figure~\ref{fig:randW} show that as $\alpha$ increases
when $\alpha<2$, there exists a near-linear relation; and
when $\alpha>2$, for $N,M$ large, the relation saturates, becoming constant, while for smaller $N,M$, there exists a near-linear relation, but with strong finite-size effects.
These numerical results demonstrate that $ \log\Vert\mathbf{W}\Vert^{2}_{F}\approx\alpha\log\lambda^{max} $ works very well for VHT random matrices, for $\alpha<2$, and that it works moderately well for MHT matrices and even some WHT matrices.
In particular, for MHT matrices, in the finite-size regime, when $N,M\sim\mathcal{O}(100-1000)$, which is typical for modern DNNs, the PL-Norm relations holds, \emph{on average}, quite well. 
This is precisely what we want in a practical engineering metric that is designed to describe average test~accuracy. 

\begin{figure}[!htb]
   \centering
   \includegraphics[scale=0.25]{img/Alpha-LogNorm-Relations.png}
   \caption{
            Numerical test of Eqn.~(\ref{eqn:basic_relation}) for random HT matrices across different HT Universality~classes.
           }
   \label{fig:randW}
\end{figure}


%% \michael{Make sure what I include in the above par is enough for what I need here.}
%% \michael{Here is the place to make explicit the connection between $\alpha$ and $\mu$, for finite $N$ and asymptotically, and to say Eqns.~(\ref{eqn:alpha_mu_vht_and_mht}) holds for both VHT and MHT, for realistic values of $N$.}

%To understand this relation better, and to sketch a proof,
%we will generate the data for a number of a Heavy-Tailed random matrices, 
%with different power law exponents $\mu$.
%Note that this linear relation holds over several log scales.  However,
%the relation does deviate from linearity at the smaller values of $\alpha\;\log_{10}\;\lambda^{max}$.
%This is readily explained below .  


%%MM%% This approximate relation formally only hold in the asymptotic limit of very small power law exponents $\alpha\rightarrow 1$ for
%%MM%% random heavy tailed matrices, but using Universality, we can safely extend it up to the finite-size MHT class, with
%%MM%% exponents $\alpha=4$ (and larger).  



%\paragraph{The PL--Norm Relation: Random Matrices versus Real Data.}
\section{The PL--Norm Relation: Random Matrices versus Real Data}
\label{sxn:appendix-random-vs-real}

\begin{figure*}[!htb]
    \centering
    \subfigure[Random Pareto Matrices] {
        \includegraphics[scale=0.25]{img/relation-rand.png} 
        \label{fig:relation-rand}
    }
    \subfigure[VGG11 Weight matrices]{
        \includegraphics[scale=0.25]{img/relation-vgg11.png} 
        \label{fig:relation-vgg11}
    }
        \caption{Relation between $\alpha\log_{10}\lambda^{max}$ and $\log_{10}\Vert\mathbf{W}\Vert^{2}_{F}$ for random (Pareto) matrices and real (VGG11) DNN weight matrices.
                 }
    \label{fig:relations}
\end{figure*}

Here, we show, numerically, that Eqn.~(\ref{eqn:basic_relation}) holds qualitatively well and 
 more generally than the special case derived previously, including into the MHT Universality class, for both random and real data.
To illustrate this, we generate a large number of HT random matrices $\mathbf{W}^{rand}(\mu)$, with varying sizes $N$ (and $Q=1$), drawn from a Pareto distribution of Eqn.~(\ref{eqn:ht_dstbn}), with exponents $\mu\in[0.5, 5]$.
We then fit the ESD of each $\mathbf{W}^{rand}(\mu)$ to a PL using the method of Clauset et al.~\cite{CSN09_powerlaw,ABP14} to obtain the empirical exponent $\alpha$.   
Figure~\ref{fig:relation-rand} shows that there is a near-perfect relation between $ \alpha\log\lambda^{max}$ and $\log\Vert\mathbf{W}\Vert^{2}_{F} $, for this random data.
We also performed a similar PL fit for VGG11 weight matrices.
(See Section~\ref{sxn:emp} for some details on the VGG11 model.)
Figure~\ref{fig:relation-vgg11} shows the results, demonstrating for the VGG11 data an increasing relation until $ \alpha\log\lambda^{max} \approx 2.5$, and a saturation after that point.
Figure~\ref{fig:relations} illustrates
(among other things\footnote{Clearly, there are also differences between the HT random and the real DNN matrices, most notably that $ \alpha\log\lambda^{max} $ achieves much larger values for the random matrices.  This is discussed in more detail in Appendix~\ref{sxn:appendix-universality}.}):
that multiplying $\alpha$ by $\log\lambda^{max}$ leads to a relation that increases linearly with the (log of the squared) Frobenius norm for HT random matrices; that the two quantities are linearly correlated for real DNN weight matrices; and that both random HT and real, strongly-correlated matrices show similar saturation effects at large PL exponents.

%% \michael{Maybe comment also about how points large on the X axis have smaller $\alpha$.}
%% 
%% \charlesX{LEAD INTO HOW WE GOT THIS SECTION... PRESENT THE PL-NORM RELATION, SHOW THAT IT IS THE RIGHT SLOPE, AND HOW THE FINITE SIZE  EFFECTS LET US EXTEND THIS RELATION ACROSS UNIVERSALITY CLASSES IN ROUGH WAY, GIVING A USEFUL METRIC FOR ENGINEERING WORK.  REAL DATA IS SHOWN.  HAS 4 PLOTS.  USES NUMERICAL METHODS DESCRIBED ABOVE.  WE MAY WANT PSEUDOCODE ALSO ?}



\section{Random Pareto versus Non-random DNN Matrices} 
%\section{Heavy-Tailed Universality: Random Pareto versus Non-random DNN Matrices} 
\label{sxn:appendix-universality}

When we use Universality, as we do in our derivation of 
the basic PL--Norm Relation, 
%Eqn.~(\ref{eqn:basic_relation}),
we would like a method that applies both to HT random matrices as well as to non-random, indeed strongly-correlated, pre-trained DNN layer weight matrices that (as evidenced by their ESD properties) are in a HT Universality class.  
To accomplish this, however, requires some care: while the pre-trained $\mathbf{W}$ matrices do have ESDs that display empirical signatures of HT Universality~\cite{MM18_TR}, they are \emph{not} random Pareto matrices.
Many of their properties, including their empirical Frobenius norms, behave very differently than that of a random Pareto matrix.  
(We saw this in Figure~\ref{fig:relations}, which showed that $ \alpha\log\lambda^{max} $ achieves much larger values for HT random matrices than real DNN weight matrices.)

To illustrate this, we generate a large number of HT random matrices $\mathbf{W}^{rand}(\mu)$, with exponents $\mu\in[0.5, 5]$, as described in Section~\ref{sxn:theory-new}.
%
We then fit the ESD of each $\mathbf{W}^{rand}(\mu)$ to a PL using the method of Clauset et al.~\cite{CSN09_powerlaw,ABP14} to obtain the empirical PL exponent $\alpha$. 
Figure~\ref{fig:fro-rand} displays the relationship between the (log of the squared) Frobenius norm and the $\mu$ exponents for these randomly-generated Pareto matrices.
(Similar but noisier plots would arise if we plotted this as a function of $\alpha$, due to imperfections in the PL fit.)
We did the same for the weight matrices (extracted from the Conv2D Feature Maps) from the pre-trained VGG11 DNN, again as described in Section~\ref{sxn:theory-new}.
Figure~\ref{fig:fro-vgg11} displays these results, here as a function of $\alpha$.

\begin{figure*}[!htb]
   \centering
   \subfigure[Random Pareto Matrices] {
      \includegraphics[scale=0.25]{img/fro-rand.png} 
      \label{fig:fro-rand}
   }
   \subfigure[Pre-trained VGG11 Weight Matrices]{
      \includegraphics[scale=0.25]{img/fro-vgg11.png} 
      \label{fig:fro-vgg11}
   }
   \caption{Dependence of Frobenius norm on PL exponents for random Pareto versus pre-trained DNN matrices.  }
   \label{fig:fnorm}
\end{figure*}

From Figures~\ref{fig:fro-rand} and~\ref{fig:fro-vgg11}, we see that the properties of $\Vert\mathbf{W}\Vert^{2}_{F}$ differ strikingly for the random Pareto versus real/ron-random DNN weight matrices, and thus care must be taken when applying these Universality principles to strongly correlated systems.
For a random Pareto matrix, $\mathbf{W}^{rand}(\mu)$, the Frobenius norm $\Vert\mathbf{W}^{rand}(\mu)\Vert^{2}_{F}$ \emph{decreases with increasing exponent} $(\mu)$; and there is a modest finite-size effect.
(In addition, as the tails of the ESD $\rho(\lambda)$ get heavier, the largest eigenvalue $\lambda^{max}$ of $\mathbf{X}$ scales with the largest element of $\mathbf{W}^{rand}(\mu)$.) 
For the weight matrices of a pre-trained DNN, however, the Frobenius norm $\Vert\mathbf{W}\Vert^{2}_{F}$ \emph{increases with increasing exponent} $(\alpha)$, saturating at $\alpha\approx 3$.
This happens because, due to the training process, the $\mathbf{W}$ matrices themselves are highly-correlated, and not random matrices with a single large, atypical element.%
\footnote{This is easily seen by simply randomizing the elements of a real DNN weight matrix, and computing the ESD~again.}
In spite of this, the ESD $\rho(\lambda)$ of these pre-trained correlations matrices $\mathbf{X}$ display Universal HT behavior~\cite{MM18_TR}.
In addition, as shown in Figure~\ref{fig:relation-vgg11}, Eqn.~(\ref{eqn:basic_relation}) is approximately satisfied, in the sense that $\alpha\log\lambda^{max}$ is positively correlated with $\log\Vert\mathbf{W}\Vert^{2}_{F} $.
This is one of the remarkable properties of Mechanistic Universality.


\section{Additional Empirical Results}
\label{sxn:appendix-addl-empirical}

In addition to the VGG and ResNet series of models, we examined a wide range of other DNNs.
Here, we summarize some of those results.

\paragraph{More Pre-trained Models.}

We present results for eleven more series of pre-trained DNN architectures, eight of which show positive results, as with the VGG and ResNet series (in Section~\ref{sxn:emp}), and three of which provide counterexample architectures.
See Table~\ref{table:models_more} for a summary of~results.

The results that perform as expected are show in
Figures~\ref{fig:models_more_0}, 
\ref{fig:models_more_1}, 
\ref{fig:models_more_2}, 
and~\ref{fig:models_more_3}.
For each set of models, our Universal metric $\hat{\alpha}$
is smaller when, for the most part, the reported (Top 1)
test accuracy is larger. 
This holds approximately
true for the three of the four DenseNet models, with
densenet169 as an outlier. 
In fact, this is the only
outlier out of 26 DNN models in these 8 architectures.
For all of the other pre-trained DNNs, smaller
$\hat{\alpha}$ corresponds with smaller test error
and larger test accuracy, as predicted by our theory.

\begin{table}[!htb]
\small
\begin{center}
\begin{tabular}{|p{1in}|c|c|c|c|c|c|c|}
\hline
Architecture 
 & Model
 & Top 1 & $\hat{\alpha} $\\
 \hline
 Working Examples & & & \\
 \hline
 DenseNet
& densenet121 & 74.43 & 1.25 \\
& densenet161 & 77.14 & 0.84 \\
& densenet169 & 75.60 & 0.68 \\
& densenet201 & 76.90 & 0.50 \\
\hline
SqueezeNet
& squeezenet\_v1\_0 & 58.69 & 2.55 \\
& squeezenet\_v1\_1 & 58.18 & 1.56 \\
\hline
CondenseNet
& condensenet74\_c4\_g4 & 73.75 & -1.83 \\
& condensenet74\_c8\_g8 & 71.07 & -1.63 \\
\hline
DPN
& dpn68 & 75.83 & 0.57 \\
& dpn98 & 79.19 & 0.11 \\
& dpn131 & 79.46 & -0.13 \\
\hline
ShuffleNet
& shufflenetv2\_wd2 & 58.52 & 5.12 \\
& shufflenetv2\_w1 & 65.61 & 2.86 \\
\hline
MobileNet
& mobilenet\_wd4 & 53.74 & 5.54 \\
& mobilenet\_wd2 & 63.70 & 4.26 \\
& mobilenet\_w3d4 & 66.46 & 4.41 \\
& mobilenet\_w1 & 70.14 & 4.19 \\
& mobilenetv2\_wd4 & 50.28 & 12.12 \\
& mobilenetv2\_wd2 & 63.46 & 4.69 \\
& mobilenetv2\_w3d4 & 68.11 & 4.21 \\
& mobilenetv2\_w1 & 70.69 & 3.50 \\
\hline
SE-ResNet
& seresnet50 & 77.53 & -0.35 \\
& seresnet101 & 78.12 & -1.24 \\
& seresnet152 & 78.52 & -1.53 \\
\hline
SE-ResNeXt
& seresnext50\_32x4d & 79.00 & 1.81 \\
& seresnext101\_32x4d & 80.04 & 0.76 \\
\hline
Counterexamples & & &  \\
\hline
ResNeXt
& resnext101\_32x4d & 78.19 & 1.22 \\
& resnext101\_64x4d & 78.96 & 1.34 \\
\hline
MeNet
& menet108\_8x1\_g3 & 56.08 & 5.31 \\
& menet128\_8x1\_g4 & 56.05 & 4.46 \\
& menet228\_12x1\_g3 & 66.43 & 4.82 \\
& menet256\_12x1\_g4 & 66.59 & 4.97 \\
& menet348\_12x1\_g3 & 69.90 & 5.74 \\
& menet352\_12x1\_g8 & 66.69 & 4.42 \\
& menet456\_24x1\_g3 & 71.60 & 5.11 \\
\hline
FDMobileNet
& fdmobilenet\_wd4 & 44.23 & 6.40 \\
& fdmobilenet\_wd2 & 56.15 & 7.01 \\
& fdmobilenet\_w1 & 65.30 & 7.10 \\
\hline
\end{tabular}
\end{center}
\caption{Results for more pre-trained DNN models.  Models provided in the OSMR Sandbox, implemented in pyTorch. Top 1 refers to the Top 1 Accuracy, which $100.0$ minus the Top 1 reported error.}
\label{table:models_more}
\end{table}


\begin{figure*}[!htb]
   \centering
   \subfigure[DenseNet] {
       \includegraphics[scale=0.30]{img/densenet-pytorch-w_alphas.png} 
       \label{fig:densenet}
   }
   \subfigure[SqueezeNet]{
       \includegraphics[scale=0.30]{img/squeezenet-pytorch-w_alphas.png} 
       \label{fig:squeezenet}
   }
   \caption{
      Pre-trained 
      Densenet and SqueezeNet PyTorch 
      Models.
      Top 1 Test Accuracy versus~$\hat{\alpha}$.
           }
   \label{fig:models_more_0}
\end{figure*}


\begin{figure*}[!htb]
   \centering
   \subfigure[CondenseNet]{
       \includegraphics[scale=0.25]{img/CondenseNet-w_alphas.png} 
       \label{fig:densenet-small}
   }
   \subfigure[DPN]{
       \includegraphics[scale=0.25]{img/DPN-w_alphas.png} 
       \label{fig:dpn-net}
   }
   \caption{
      Pre-trained 
      CondenseNet and DPN
      Models.
      Top 1 Test Accuracy versus
      $\hat{\alpha}$.
           }
   \label{fig:models_more_1}
\end{figure*}


\begin{figure*}[!htb]
   \centering
   \subfigure[ShuffleNet]{
      \includegraphics[scale=0.25]{img/ShuffleNet-w_alphas.png}
      \label{fig:shufflenet-small}
   }
   \subfigure[MobileNet]{
      \includegraphics[scale=0.25]{img/MobileNet-w_alphas.png} 
      \label{fig:resnet-small}
   }
   \caption{
      Pre-trained 
      ShuffleNet and MobileNet
      Models.
      Top 1 Test Accuracy versus
      $\hat{\alpha}$.
           }
   \label{fig:models_more_2}
\end{figure*}


\begin{figure*}[!htb]
   \centering
   \subfigure[SeResNet]{
      \includegraphics[scale=0.25]{img/SeResNet-w_alphas.png}
      \label{fig:shufflenet-small}
   }
   \subfigure[SeResNeXt]{
      \includegraphics[scale=0.25]{img/SeResNeXt-w_alphas.png}
      \label{fig:shufflenet-small}
   }
   \caption{
      Pre-trained 
      SeResNet and SeResNeXt
      Models.
      Top 1 Test Accuracy versus
      $\hat{\alpha}$.
           }
   \label{fig:models_more_3}
\end{figure*}


\paragraph{Counterexamples.}
In such a large corpus of DNNs, there are of course exceptions for a predictive theory.
See
Figure~\ref{fig:counter-examples}
(as well as the corresponding rows of Table~\ref{table:models_more})
for the counterexamples.
These are ResNeXt, MeNet, and FDMobileNet.
For ResNeXt, there are only two models, and the $\hat{\alpha}$ is larger
for the less accurate model. 
For MeNet, there are seven different models,
and there is no discernible pattern in the data. 
Finally, for FDMobileNet,
there are three different pre-trained models, and, again, the 
$\hat{\alpha}$ is larger for the less accurate models. 
We have not looked in detail at these results and simply present them for completeness. 
 
\begin{figure}[!htb]
   \centering
   \subfigure[ResNeXt]{
      %\includegraphics[scale=0.19]{img/ResNeXt-w_alphas.png} 
      \includegraphics[scale=0.22]{img/ResNeXt-w_alphas.png} 
      \label{fig:resnet-small}
   }
   \subfigure[MeNet]{
      %\includegraphics[scale=0.19]{img/MeNet-w_alphas.png} 
      \includegraphics[scale=0.22]{img/MeNet-w_alphas.png} 
      \label{fig:menet-net}
   }
   \subfigure[FDMobileNet]{
      %\includegraphics[scale=0.19]{img/FDMobileNet-w_alphas.png} 
      \includegraphics[scale=0.22]{img/FDMobileNet-w_alphas.png} 
      \label{fig:resnet-small}
   }
   \caption{
      Pre-trained 
      ResNeXt, MeNet, and FDMobileNet
      Models 
      provide counterexamples to our main trends.
      Top 1 Test Accuracy versus
      $\hat{\alpha}$.
           }
   \label{fig:counter-examples}
\end{figure}


\section{Additional Discussion}
\label{sxn:appendix-addl-discussion}

We have presented an \emph{unsupervised} capacity control metric which predicts trends in test accuracies of a trained DNN---without peeking at the test data. 
This complexity metic, $\hat{\alpha}$ of Eqn.~(\ref{eqn:alpha_hat_specific}), is a weighted average of the PL exponents $\alpha$ for each layer weight matrix, where $\alpha$ is defined in the recent HT-SR Theory~\cite{MM18_TR}, and where the weights are the largest eigenvalue $\lambda^{max}$ of the correlation matrix $\mathbf{X}$.  
%
We examine several commonly-available, pre-trained, production-quality DNNs by plotting $\hat{\alpha}$ versus the reported test accuracies.
This covers classes of DNN architectures including the VGG models, ResNet, DenseNet, etc. 
In nearly every class, and except for a few counterexamples, smaller $\hat{\alpha}$ corresponds to better average test accuracies, thereby providing a strong predictor of model quality.
%
We also show that this new complexity metric $\hat{\alpha}$ is approximately the average log of the squared Frobenius norm of the layer weight matrices, $\langle\log\Vert\mathbf{W}\Vert_{F}^{2}\rangle$, when accounting for finite-size effects:
$$
 \alpha\log\lambda^{max}\approx\log\Vert\mathbf{W}\Vert^{2}_{F}  .
$$
This provides an interesting connection between the Statistical Physics approach to learning (from Martin and Mahoney~\cite{MM17_TR,MM18_TR}, that we extend here) and methods such as that of Liao et al.~\cite{LMBx18_TR}, who use norm-based capacity control metrics to bound worst-case generalization~error.

%% We should mention two higher-level comments.
%% %
%% First, 
%% one of the main insights of our approach is to highlight the importance of what might seem to be a technical issue to ignore, but which in our experience is \emph{extremely} important: the scaling or normalization for weight matrices used in the DNN, and how this relates to the difference between finite-size versus asymptotic effects.
%% This scaling issue has been highlighted perhaps most recently by Bartlett et al.~\cite{BFT17_TR} and Liao et al.~\cite{LMBx18_TR}.
%% The latter were interested in showing that classical generalization bounds can be tight---when normalization is performed appropriately.
%% Our approach complements theirs; but, to our knowledge, our approach is the first to highlight the connection with finite-size effects.
%% %
%% Second, 

It is worth emphasizing that 
we are taking a 
very %somewhat 
non-standard approach (at least for the DNN and ML communities) to address our main question.
We did not train/retrain lots and lots of (typically rather small) models, analyzing training/test curves, trying to glean from them bits of insight that might then extrapolate to more realistic models.
Instead, we took advantage of the fact that there already exist many (typically rather large) publicly-available pre-trained models, and we analyzed the properties of these models.
That is, we viewed these publicly-available pre-trained models as artifacts of the world that achieve state-of-the-art performance in computer vision, NLP, and related applications; and we attempted to understand why.
To do so, we analyzed the empirical (spectral) properties of these models; 
%from this, we formed a hypothesis as to why they perform well; 
and we then extracted data-dependent metrics to predict their generalization performance on production-quality models.
Given well-known challenges associated with training, 
%and given our results here as well as other recent results~\cite{MM18_TR},
we suggest that this methodology be applied more generally.

Finally, one interesting aspect of our approach is that we can apply these complexity metrics \emph{across related DNN architectures}. 
This is in contrast to the standard practice in ML.
The equivalent notion would be to compare margins across SVMs, applied to the same data, but with different kernels. 
One loose interpretation is that a set of related of DNN models (i.e., VGG11, VGG13, etc.) is analogous to a single, very complicated kernel, and that the hierarchy of architectures is analogous to the hierarchy of hypothesis spaces in more traditional VC theory.
%\charlesX{more here ?  like this ?}
%\michael{Let's discuss this, to see what we can squeeze, given what is now popular.}   
Making this idea precise is clearly of interest.

We expect our result will have applications in the fine-tuning of pre-trained DNNs used for transfer learning, as in NLP and related applications.
Moreover, because we do not need to peek at the test data, our approach may prevent information from leaking from the test set into the model, thereby helping to prevent overtraining and making fined-tuned DNNs more robust.
Finally, our work also leads to a much harder theoretical question: is it possible to characterize properties of realistic DNNs to determine whether a DNN is overtrained---without peeking at the test data?  
 

{ \iffalse

\newpage
\section{Appendix: Derivation of two relations}
\label{sxn:appendix-derivation-two-relations}

Here we derive the two relations, Eqns.~(\ref{eqn:alpha_mu_vht_and_mht}) and~(\ref{eqn:scaling_of_lambda_max}).
\michael{Maybe do this later.}

\fi }


{ \iffalse 

\newpage
\section{Appendix: Michael's Longer Derivation}
\label{sxn:appendix-michael_derivation}

\michael{MM probably remove this entirely, unless relate to synthetic empirical results.}

Here, we derive an expression for the ratio of the log of the Frobenius norm of $W$ to the log of the spectral norm of $W$.
Once we settle on presentation, with normaliation, etc., this will probably be a ``subroutine'' in our analysis.
To simplify things, we will be interested in the matrix $W$, and in particular the Frobenius norm $\|W\|_F$ and spectral norm $\|W\|_2$ of this matrix.
Then, given these expressions, we will derive other things, e.g., norms of correlation matrices with different normalizations, etc., by using different normalizations.  

Recall that we are modeling the matrix $W$ as a random matrix with heavy-tailed entries.
XXX.  WE MIGHT WANT A FIGURE SHOWING THE REAL DATA LOOKS LIKE THIS, LIKE THE BURDA PAPER.
Thus, the Frobenius norm is going to be related to the second moment of the entries, and the spectral norm is going to be related to the largest entry.
XXX.  CITE AUFFINGER PAPER FOR EXACTLY THE RANGE OF VALIDITY OF THIS.
An important issue will be the power law exponent, since---depending on it---familiar results will hold or will fail to hold.
For the moment, let's ignore that we are dealing with matrices, and let's focus on drawing elements from a heavy-tailed distribution, and computing various moments and extreme values of empirical draws.

Consider the extreme case of a HT distribution, namely a PL distribution.
XXX.  INCLUDE SOMETHING ABOUT SLOWLY VARYING FUNCTION AS WELL AS XMIN VALUE.
Up to a slowly-varying function, the general form of the probability distribution function is
$$
p(x) = \frac{C}{x^{1+\mu}}  = C x^{-1-\mu} , 
$$
where $\mu > -1$, and where $x \in [x_{min},\infty)$.
The cdf 
XXX ACTUALLY ONE MINUS THAT 
is then
$$
P_{\ge}(x) = \int_x^{\infty} p(x^{\prime}) dx^{\prime} 
           = \frac{C}{\mu} \frac{1}{x^{\mu}}  
           = \frac{C}{\mu} x^{-\mu}  .
$$
In order to compute $C$ and normalize these expressions, let
\begin{eqnarray*}
1 = \int_{x_{min}}^{\infty} p(x) dx 
  = C \int_{x_{min}}^{\infty} x^{-1-\mu} dx 
  = \frac{C}{-\mu} x^{-\mu} |_{x_{min}}^{\infty}  
  = \frac{C}{\mu} x_{min}^{-\mu}   ,
\end{eqnarray*}
which is valid (i.e., the integral converges and exists) if $\mu > 0$.
From this, is follows that the normalization constant is
\begin{equation}
C = \mu x_{min}^{\mu}  .
\label{eqn:pl_normalization}
\end{equation}
Thus, if $\mu > 0$, then the probability distribution function is 
\begin{equation}
p(x) 
%     = \frac{\mu}{x_{min}}\left( \frac{x_{min}}{x}\right)^{1+\mu}  
     = \frac{\mu}{x_{min}}\left( \frac{x}{x_{min}}\right)^{-1-\mu}  ,
\label{eqn:pl_pdf}
\end{equation}
and the cdf 
XXX ACTUALLY ONE MINUS THAT
is 
\begin{equation}
P_{\ge}(x) 
%           = \left( \frac{x_{min}}{x} \right)^{\mu}  
           = \left( \frac{x}{x_{min}} \right)^{-\mu}  .
\label{eqn:pl_one_minus_cdf}
\end{equation}
XXX.  MENTION SLOWLY VARYING THING, MAYBE AS CLAUSET DID.

An important aspect of heavy-tailed probability distributions is that extreme values, i.e., values very far from the mean (when the mean is even defined) are not extremely uncommon (as they are for distributions in the Gaussian universality class).
Of particular relevance for us is the largest value $x_{max}$ obtained when sampling from Eqn.~(\ref{eqn:pl_pdf}) in $n$ i.i.d. trials.
It is known, see e.g.~\cite{SornetteBook,BouchaudPotters03,newman2005_zipf}, that the expectation of $x_{max}$ for $\mu\in(1,2)$ 
XXX IS THIS TRUE FOR MORE GENERAL PL PARAMETERS
is given~by:
$$
\ExpectBracket{x_{max}} \approx x_{min} n^{1/\mu}  .
$$

Let's return to the expression given in Eqn.~(\ref{eqn:pl_pdf}) and compute the first few moments of this distribution.
The first moment is
\begin{eqnarray*}
\ExpectBracket{x} = \int_{x_{min}}^{\infty} x p(x) dx  
                  = C \int_{x_{min}}^{\infty} x^{-\mu} dx 
                  = \frac{C}{1-\mu} x^{1-\mu} |_{x_{min}}^{\infty} 
                  = \frac{\mu}{\mu-1} x_{min}    ,
\end{eqnarray*}
which is valid if $\mu > 1$.
Similarly, the second moment is
\begin{eqnarray*}
\ExpectBracket{x^2} = \int_{x_{min}}^{\infty} x^2 p(x) dx  
                    = C \int_{x_{min}}^{\infty} x^{1-\mu} dx 
                    = \frac{C}{2-\mu} x^{2-\mu} |_{x_{min}}^{\infty} 
                    = \frac{\mu}{\mu-2} x_{min}^2  ,  
\end{eqnarray*}
which is valid if $\mu > 2$.
While this second moment expression is valid for $\mu > 2$, we are going to want a similar expression for $\mu \in (1,2)$.
For this we can integrate up to $x_{max}$, rather than up to $\infty$.
In more detail, for $\mu \in (1,2)$, the empirical second moment is
\begin{eqnarray*}
\ExpectBracket{x^2} &=&       \int_{x_{min}}^{x_{max}} x^2 p(x) dx  \\
                    &=&       \frac{C}{2-\mu} x^{2-\mu} |_{x_{min}}^{x_{max}}  \\
                    &\approx& \frac{\mu}{2-\mu} x_{min}^{\mu} \left( x_{min}^{2-\mu} n^{(2-\mu)/\mu} - x_{min}^{2-\mu} \right) \\
                    &\approx& \frac{\mu}{2-\mu} x_{min}^{2} n^{(2-\mu)/\mu}   ,
\end{eqnarray*}
which is valid for $\mu\in(1,2)$.
Note that in these expressions and the expressions below, we follow previous work \cite{MM18_TR} and don't compute expressions for $\mu=2$ precisely.
(They are known to lie in yet another universality class~\cite{SornetteBook,BouchaudPotters03}, and we don't expect to resolve the difference numerically.)

Finally, consider an $N \times N$ matrix $W$. 
Then, 
$\|W\|_2 = w_{min} N^{2/\mu}$ (for $\mu\in(1,2)$) and
$\|W\|_2 = w_{min} N^{1/2}$ (for $\mu>2$).
XXX.  CHECK THAT I AM NOT OFF By A FACTOR OF N.
Thus, for the spectral norm, we have that: 
\begin{equation}
\|W\|_2^2 = \left\{ \begin{array}{ll}
                       w_{min}^2 N^{4/\mu} & \mbox{if $\mu\in(1,2)$} \\
                       w_{min}^2 N & \mbox{if $\mu > 2$} (XXX CHECK)
                    \end{array}
            \right.
\end{equation}
Similarly, for the Frobenius norm, we have that:
\begin{equation}
\|W\|_F^2 = \left\{ \begin{array}{ll}
                      \frac{\mu}{2-\mu} w_{min}^2 N^{(4-2\mu)/\mu} & \mbox{if $\mu\in(1,2)$} \\
                      \frac{\mu}{\mu-2} w_{min}^2 N^2 & \mbox{if $\mu > 2$} 
                    \end{array}
            \right.
\end{equation}

We are interested in the function of $\mu$ defined as:
$$
f = f(\mu) = \frac{\log \|W\|_F^2}{\log \|W\|_2^2}  .
$$
From the above, if we take logs, then for $\mu\in(1,0)$, we get:
$$
f(\mu) = \frac{ \log\left(\frac{\mu}{2-\mu}\right) + \log(w_{min}^2) + \frac{4}{\mu}\log N - 2 \log N }{ \log(w_{min}^2) + \frac{4}{\mu}\log N }
$$
and for $\mu > 2$, we get:
$$
f(\mu) = \frac{ \log\left(\frac{\mu}{\mu-2}\right) + \log(w_{min}^2) + 2 \log N }{ \log(w_{min}^2) + \log N }
$$

\fi }


{ \iffalse 

\newpage
\section{Appendix: Refs: MM TO INCORPORATE THESE INTO TEXT}

Our theory of Implicit Self Regularization used Heavy Tailed Random Matrix Theory (HT RMT), and here we use HT RMT also~\cite{MM18_TR}.
See also our prior results on \cite{MM17_TR}.

\cite{NTS14_TR} is 
Neyshabur et al. on :
``In search of the real inductive bias: on the role of implicit regularization in deep learning''

\cite{NTS15} is
Neyshabur et al. on:
``Norm-Based Capacity Control in Neural Network''

\cite{NBMS17_TR} is 
Neyshabur et al. on:
``Exploring generalization in deep learning''

\cite{AGNZ18_TR} is 
Arora et al. on:
``Stronger generalization bounds for deep nets via a compression approach''

\cite{ACH18_TR} is 
Arora et al. on:
``On the Optimization of Deep Networks: Implicit Acceleration by Overparameterization''

\cite{Bar97} is
Bartlett on:
``For valid generalization, the size of the weights is more important than the size of the network''

\cite{BFT17_TR} is 
Bartlett et al. on:
``Spectrally-normalized margin bounds for neural networks''

\cite{SHNx17_TR} is 
Soudry et al. on: 
``The implicit bias of gradient descent on separable data''

\cite{YM17_TR} is 
Yoshida and Miyato on:
``Spectral norm regularization for improving the generalizability of deep learning''

\cite{LMBx18_TR} is 
Liao et al. on:
``A surprising linear relationship predicts test performance in deep networks''

\cite{PLMx18_TR}
is Poggio el at. on: 
``Theory {IIIb}: Generalization in Deep Networks''

\cite{KKB17_TR} is 
Kawaguchi et al. on:
``Generalization in Deep Learning''

\cite{NBS17_TR} is:
Neyshabur et al. on: 
``A {PAC}-{B}ayesian Approach to Spectrally-Normalized Margin Bounds for Neural Networks''

\cite{ZF18_TR} is
Zhou and Feng on:
``Understanding Generalization and Optimization Performance of Deep {CNN}s''

\cite{BJNx01_TR} is
Burda et al. on:
``{L}{\'e}vy Matrices and Financial Covariances''

\cite{MN09_TR} is
Mahoney and Narayanan on:
``Learning with Spectral Kernels and Heavy-Tailed Data''
 
\fi }



\end{document}

